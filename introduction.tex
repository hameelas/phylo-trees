\section{Introduction}

Phylogenetic trees are biological applications of clustering that provide evolutionary relations between sets of DNA or protein sequences. A rooted phylogenetic tree is a directed tree where the root is the least common ancestor of the leaves. In contrast, unrooted phylogenetic trees describe the relationship between the leaves without defining or assuming an ancestor. For this project, we will specifically address unrooted phylogenetic trees. Each node in the tree represents the most recent common ancestor of its children and the weights on the branches of the tree represent the difference between the sequences.

Phylogenetic trees provide insight and visualization of evolution and genetic mutation. This can be helpful for painting a better picture of how different sets of DNA or protein sequences are related, classifying species such as Apocynaceae \cite{sennblad2002}, and assessing the targets of microbial warfare \cite{riley2003}. Furthermore, specific phylogenetic trees, such as the Y-Chromosomal phylogenetic tree, have applications in forensic science \cite{van2013}. Phylogenetic trees can also be used to find the origin of pathogen outbreaks, like the avian influenza virus H7N3 outbreaks \cite{lu2014}. Moreover, they have applications in comparative linguistics to find how words are adopted from one language to another. Recently, a linguistic phylogenetic approach has been used for describing the evolution of color terms in languages \cite{haynie2016}.That is why it is desirable to have an efficient method of constructing phylogenetic trees from sets of DNA or protein sequences.

In this project, we will be utilizing the Neighbor Joining algorithm to create a phylogenetic trees given a distance matrix, which describes the difference between sequences, as input. The algorithm was purposed by Naruya Saitou and Masatoshi Nei in 1987 \cite{saitou1987neighbor} and it has a time complexity of $\mathcal{O}(n^3)$ where n is the size of the set of sequences. We will also describe in this paper an attempt we had to provide a speedup to the algorithm.
