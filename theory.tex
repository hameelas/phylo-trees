\section{Theoretical Work}

Our approach to speedup the algorithm is to simulate the changes that occur to matrix $Q$ during the algorithm in a data structure which handles the queries in polylogarithmic time. The desired data structure should capture changes to $Q$ after each step while maintaining the minimum element in $Q$ for the next step. We can use the minimum element in $Q$ to merge two taxa, and make queries to perform the induced changes in $Q$.

\emph{Segment Tree} is useful data structure to implement interval find and modification queries. One can extend the notion of the segment tree into higher dimensions, say a $d$-dimensional segment tree, by simply creating a one-dimensional segment tree of $(d-1)$-dimensional segment trees. This results in query running time of $O(n^{d-1}\log{n})$ for submatrix find and modification queries. Ibtehaz et. al~\cite{ibtehaz2018multidimensional} showed it is possible to implement a multi-dimensional segment tree capable of handling find and modification queries in $O(\log{n}^d)$ time for a various range of aggregate operators.

[TBD]